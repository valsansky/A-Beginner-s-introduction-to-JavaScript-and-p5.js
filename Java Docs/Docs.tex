% NOTES FOR FUTURE WORK:
%  Check \Extras\Notes for future work.txt

\documentclass[11pt, a4paper]{article}
\usepackage{graphicx} % images
\graphicspath{{Assets/}}
\usepackage{listings} % for code examples

% clickable titles
\usepackage[colorlinks=true,linkcolor=black,
            anchorcolor=black,citecolor=black,
            filecolor=black,menucolor=black,
            runcolor=black,urlcolor=black]{hyperref}

%% exercise boxes
    \usepackage{tcolorbox}
    \usepackage{enumitem}

    \newcounter{exercise}[section] % creates a counter for exercises
    \renewcommand{\theexercise}{\thesection.\arabic{exercise}} % formats exercise number; e.g. 2.1

    \newenvironment{exercise}[1][]{ % Takes #1:title
        \refstepcounter{exercise} 
        \begin{tcolorbox}[title=Exercise \theexercise: #1]
    }{
        \end{tcolorbox}
    }

%% practices
    \newcounter{practice}[section]
    \renewcommand{\thepractice}{\thesection.\arabic{practice}}

    \newenvironment{practice}[1][]{
        \refstepcounter{practice}
        \paragraph{Practice \thepractice} #1
    }{}
%% Mini projects
    \newcounter{project}
    \renewcommand{\theproject}{\arabic{project}}

    \newenvironment{project}[1][]{ 
        \refstepcounter{project}
        \begin{center}
            \textbf{Project \theproject: #1}
        \end{center}
    }{}

\title{A beginner’s introduction to JavaScript 
            
and p5.js}
% \author{Andreas Vorgod Damgaard, Platon Arturovich Kir'yanov and Valdemar Sander Skytte}
\author{Europe Connexion's team\thanks{Made in collaboration with Academia Main().}}
\date{\today}

\begin{document}
\maketitle

% p5 logo image
\begin{figure}[htp]
    \centering
    \includegraphics[width=4cm]{P5js_Logo.png}
    \caption{p5.js logo}
    \label{fig:p5js_Logo}
\end{figure}

\hfill

% abstract
\begin{abstract}
    This document is meant to help beginners and intermediates learn 
    how to write code in JavaScript using the p5 library. The p5 library is 
    a library optimized for beginners, that makes it easy and intuitive to 
    draw and create simple programs. p5 uses JavaScript and is therefore 
    a great way to learn the language and builds a strong foundation for 
    learning other languages in the future. By the end of this document 
    the reader should have a solid grasp on the syntax of the language, as 
    well as how to operate within it using the p5.js library.
\end{abstract}

\clearpage

\tableofcontents

\section{Introduction to p5.js}
    \subsection{What is p5.js?}
        P5.js is a library for JavaScript. It is optimized for learning to code, and drawing easily.
        It is a ideal way to start learning the basics of coding. You learn a lot about JavaScript, and
        build a great foundation for learning other programming languages in the future.

        Use https://p5js.org/reference/ to look up functions, and see all the build in p5 functions.
    \subsection{The foundation of p5.js}
        P5.js works using two primary functions:
        \begin{enumerate}
            \item function setup()
            \item function draw()
        \end{enumerate}

        The setup function is used for initializing the document. This typically involves creating a canvas.
        The canvas will be the foundation for your code, here you can draw a lot of different things that's
        being controlled by the program. The canvas works like a coordinate system, with an x- and y-axis.
        The origin is in the top left corner.

        \begin{figure}[htp]
            \centering
            \includegraphics[width=8cm]{canvas.png}
            \caption{canvas in p5.js with coordinate system}
            \label{fig:canvas}
        \end{figure}

        The draw function will run each frame. The draw function is necessary to, for example, code animations.
        The draw function could for example code a circle bouncing between walls of the canvas.

    \subsection{Getting started}
        We will walk you through getting ready to code in this chapter. Otionally, when
        you're done setting everything up, check out the section about GitHub.

        \subsubsection{Visual Studio Code}
            When coding, we recommend using Visual Studio Code (VS Code).
            VS Code is a free, open-source code editor that is very effecient, and has a lot of features.

            You can download VS Code at: https://code.visualstudio.com/download 

            \begin{center}
                \textbf{Understanding The VSCode Interface}
            \end{center}

            \begin{figure}[htp]
                \centering
                \includegraphics[width=12.5cm]{VSCode_Interface.png}
                \caption{VSCode interface}
                \label{fig:VSCode_Interface}
            \end{figure}

            \noindent The layout is very simple and modular. On the left-hand side you 
            will notice a few icons. These will be explained below in further 
            detail.
            \begin{itemize}
                \item Starting from the top, there is the paper-icon, called the Explorer. 
                    This tab allows you to view and manage your project files and folders.
                \item The four next icons are respectively Find \& Replace, Source control (more 
                    on that later), Run \& debug and the extension tab.
                \item At the bottom, you will find the Accounts and settings tab.
            \end{itemize}

            \begin{center}
                \textbf{Essential Extensions}
            \end{center}
            VSCode supports different extensions, that in many ways can improve your workflow.
            Below are three extensions we highly recommend you get:

            \paragraph{P5 Project Creator \textit{(by Ultamatum)}} This extension makes it easier 
            to create default p5.js projects in VS Code by generating a ready-to-use 
            project structure with all the necessary files, which saves time compared to 
            doing it manually.
            \paragraph{p5js Snippets \textit{(by Acidic)}} This extension provides snippets for all 
            functions in p5.js, essentially shortcuts, which elevate the coding 
            experience by speeding up the process of coding whilst also reducing typing 
            errors.
            \paragraph{Live Server \textit{(by Ritwick Dey)}} This extension simply allows you to 
            launch a local development server in your browser, which automatically refreshes when 
            you save changes in the code editor. It's a very effecient and easy way to run your code
            locally.
            \\

            To install an extension, first go the extensions tab, and search for the extension.
            Find and select the correct extension (check the author), then click install.

        \subsubsection{Creating A New Project}
            If you installed P5 Project creator before, creating a new project is very simple:
            \begin{enumerate}
                \item Create a project folder on your computer
                \item Open VSCode and Press \textit{File} in the top left corner, and select "Open Folder"
                \item Find and select your project folder
                \item Press view in the top bar of VSCode, and then press command palette
                \item Write "Create P5 Project" and press enter.
                \item Done!
            \end{enumerate}

            The extension will create two files:
            \begin{itemize}
                \item Index.html
                \item Script.js
            \end{itemize}

            The file names (except file extension, of course!) can be changed, but these are the standard names.
            You will need both files to run your project. 
            You will be writing your code inside the JavaScript file. The HTML file will only be used to load 
            everything from the JavaScript file.
            \\

            If you want to setup the files manually, still create a project folder. Then create the two files
            (Index.html and script.js) manually by right clicking inside your folder on VSCode and then select
            the option "\textit{New File}".
            When you created both files paste the following code into your HTML and JavaScript file respectively.
            \\

            \textbf{HTML Template}

            \begin{lstlisting}
    <!DOCTYPE html>
    <html lang="en">
    <head>
        <meta charset="utf-8">
        <title>Exercise 1</title>

        <script src="https://cdnjs.cloudflare.com/
            ajax/libs/p5.js/1.4.0/p5.js"></script>
    </head>

    <body>
        <script src="sketch.js"></script>
    </body>

    </html>
            \end{lstlisting}

            \textbf{JavaScript Template}
            \begin{lstlisting}
    function setup() {
        createCanvas(400, 400);
    }

    function draw() {

    }
            \end{lstlisting}

        \subsubsection{Running the code}
            Now that the necessary files are set up, you can try running the file. 
            This is done by either right-clicking on the index.html file in the Explorer 
            tab and selecting the option “Open with Live Server”, or simply by opening 
            the index.html file and clicking the “Go Live”-button in the bottom right 
            corner. This should open your browser, where you will be able to view your 
            program. 
            \\
            Try adding the function \textit{background(220);} inside the draw function.
            When you open the browser with your project you should see a lightgrey box.
            This is your canvas! Remember to save your project to see it.

\section{Variables}
    \subsection{What Is a Variable?}
        A variable acts as a container, that saves a value in the memory. You can 
        think of it as a box with a name written on it. When we use the name, we 
        get access to the content.
        
        In JavaScript, creating a variable is typically done using “let”.
        \begin{lstlisting}
        let x = 100;
        \end{lstlisting}
        When creating this variable, three things happen:

        \begin{enumerate}
            \item We create a variable \textbf{\textit{(let)}}
            \item We give it a name \textbf{\textit{(x)}}
            \item We give it a value \textbf{\textit{(100)}}
        \end{enumerate}

        It's important to note that choosing meaningful names for your variables 
        can help a lot in your project. E.g.

        \begin{lstlisting}
        let playerX = 100;
        let speed = 5;
        let score = 0;
        \end{lstlisting}
        
        It makes your code easier to read and understand.
    
    \subsection{Why Are Variables Important?}
        In p5.js your canvas is updated several times per second in the draw() function
        (around 60 times, depending on the framerate). If we want something 
        to move or change, we must save values in variables to change them continuously.
        
        \begin{lstlisting}
        let x = 50;

        function draw() {
            x = x + 1;
        }
        \end{lstlisting}

        In this example, we are constantly incrementing the variable x’s value
        by 1. If a circle was drawn at the x value, we can imagine that the circle 
        will move 1 pixel to the right each frame.
    
    \clearpage % Temporary Formatting

    \subsection{Datatypes in p5.js}
        Variables can contain a lot of different values. These values could be text, 
        numbers and so on. The type of data you assign to the variables, are called 
        datatypes.
        
        Here is a list of possible data types, that you can assign to a variable:

        \begin{center}
            \begin{tabular}{p{2.9cm}|p{4cm}|p{4.5cm}}
                \textbf{Datatype} & \textbf{Explanation} & \textbf{Example} \\ [0.5ex]
                \hline
                Integer \textit{(number)}  & An integer is a whole number. & let age = \textbf{18};\\ 
                \hline
                Float \textit{(number)} & A float is a decimal number. & let height = \textbf{180.5}; \\
                \hline
                String & A string is text. To make a string the value should be surrounded by quotation marks. & let name = \textbf{"erasmus"}; \\
                \hline
                Boolean & A Boolean is a binary value; it can either be True or False. & let hasGirlfriend = \textbf{false}; let working = \textbf{true}; \\
            \end{tabular}
        \end{center}

        Note that JavaScript does not distinguish between Integers and floating-
        point numbers. Both types are classified as \textit{number}.
        
        Different datatypes can be added together. This will often affect the 
        output data type. Try the following small exercise. Dont worry if you get it 
        wrong, it can be a bit confusing.


        \begin{exercise}[Guess the output datatype]
            \begin{enumerate}[label=\alph*)]
                \item let output = age + height;
                \item let output = height + name;
                \item let output = name + hasGirlfriend;
            \end{enumerate}
        \end{exercise}

    \subsection{Arithmetic Operations in p5.js}
        Arithmetic operations are mathematical operators, and are used to manipulate 
        numbers.

        \begin{center}
            \begin{tabular}{c | c}
                \textbf{Operator} & \textbf{Meaning} \\
                \hline
                + & Addition \\
                \hline
                - & Subtraction \\
                \hline
                * & Multiplication \\
                \hline
                / & Division \\
                \hline
                \% & Modulus \\
            \end{tabular}
        \end{center}

        The operators can be used on their own, and will evaluate two elements. 
        E.g. two variables.
        \begin{lstlisting}
        let a = 8;
        let b = 3;

        let sum = a + b;
        let difference = a - b;
        let product = a * b;
        let quotient = a / b;
        let rest = a % b;
        \end{lstlisting}

        Using the operators on their own will not affect the variables value. If 
        you add an equal sign \textit{(=)} after any of the operators, the variable value would be updated E.g.

        \begin{lstlisting}
        let score = 0; 

        score += 10; // adds 10 to the variable 
                            value
        \end{lstlisting}
        \textit{Comments in JavaScript are written as "//"}

        In this code \textit{score} changes from 0 to 10.

        For incrementing a variable value with 1, use ++

        For decrementing a variable value with 1, use - - 
        E.g.

        \begin{lstlisting}
        let score = 5;

        score++; // score increases to 6
        score--; // score decreases back to 5;
        \end{lstlisting}

        \subsubsection{What Is Modulus \textit{(\%)}?}
            Modulus \textit{(\%)} returns the remainder when dividing one number by another. 
            For example when dividing 10 by 3, you will get a remainder of 1.

            Modulus can for example be used to check if a number is even or uneven:

            \begin{lstlisting}
            if (number % 2 = = = 0) {
                console.log("Number is even");
            }
            \end{lstlisting}

            This code prints “Number is even” to the console if a number defined by 
            the \textit{number} variable is even. Any number that is even will when divided by 
            two, leave a rest of zero. Therefore, a number must be even, if it’s rest of 
            division by 2, is equal to zero.

    \subsection{\textbf{\textit{Coding practice: Variables}}}
        \begin{practice}
            Assign a variable “x” to the value 10 and print its value to the console.
        \end{practice}
        \begin{practice}
            Create two variables a = 8 and b = 2. Then print the sum, difference, 
            product, quotient and the remainder of division (using modulus: \%) of the two numbers.
        \end{practice}
        \begin{practice}
            Draw a circle in the middle of the canvas with a diameter defined by a 
            variable d(choose any value, for example 100). Use the text() function with 
            the variable, to write the diameter inside the circle. 
            
            Updating d, should update the diameter and the text displayed.
        \end{practice}
        \begin{practice}
            Make a mouse-tracker that displays relevant information about your mouse on 
            canvas. Use mouseX and mouseY to write down the mouses current position, 
            pmouseX and pmouseY to write down the mouses previous position. Then create 
            two new variables, velocityX and velocityY that describes the mouses speed in 
            x and y.
        \end{practice}
        
\section{Conditional Execution}
    In this chapter we will learn how to make decisions in our programs using 
    conditional execution. Up till this point, our code has run line by line, the 
    same way every time. With conditional execution, we can make the program react 
    differently depending on conditions. 

    \subsection{What Is Conditional Execution?}
        Conditional execution means:	
        \begin{itemize}
            \item Run this code ONLY if something is true
        \end{itemize}
        
        In JavaScript we use the "if-statement"

        \begin{lstlisting}
        if(condition) {
            // code runs if condition is true
        }
        \end{lstlisting}

        The condition statement must be true for the execution to take place.

        \begin{lstlisting}
        let age = 20;

        if (age >= 18) {
        console.log("You are an adult");
        }
        \end{lstlisting}

        Here we can see that the condition 20$>$=18 is true, and therefore "You 
        are an adult" will be printed to the console. The operator "$>$=" is called a 
        Comparison operator, that means greater than or equal to. If age was less 
        than 18, nothing would be printed, because the condition was not met.

    \subsection{Comparison Operators}
        Comparison operators are used for comparisons. They always return a 
        Boolean value (true/false), based on the comparison.
        \begin{center}
            \begin{tabular}{p{2.9cm}|p{4cm}|p{4.5cm}}
                \textbf{Symbol} & \textbf{Operation} & \textbf{Explanation} \\
                \hline
                == & Equal & Checks if two values are equal\\
                \hline
                === & Strict equal & Checks if two values and their types are equal\\
                \hline
                != & Not equal & Checks if two values are equal, returns true if they are not\\ 
                \hline
                !== & Strict not equal & Checks if two values or their datatypes are equal, returns true if any of these are not\\
                \hline
                $>$ & Greater than & Checks if a value is greater than another\\
                \hline
                $<$ & Lesser than & Checks if a value is lesser than another\\
                \hline
                $>$= & Greater than or equal & Checks if a value is greater than or equal to another\\
                \hline
                $<$= & Lesser than or equal & Checks if a value is lesser than or equal to another\\

            \end{tabular}
        \end{center}

    \subsection{if...else Statements}
        When we check if a condition is true or false, it can be useful to have one 
        thing happen if it's true, and another thing happen if it's false. This is done 
        using if...else statements.

        \begin{lstlisting}
        if(condition) {
            // runs if true
        } else {
            // runs if false
        }
        \end{lstlisting}

        Example of usage:

        \begin{lstlisting}
        let temperature = 17;

        if (temperature > 20) {
            console.log("It is hot");
        } else {
            console.log("It is cold");
        }
        \end{lstlisting}

        Only one of the "blocks" will run. In this case the else-statement will 
        run, and print "It is cold", because the condition \textit{17 $>$ 20} will return false.

        \subsubsection{else if statement}
            The previous example with temperature, was pretty naive, since it only had 
            two options. The example would say that the weather was cold even if it was 
            19 degrees, which does not make sense for a lot of people. We can work 
            around this by adding more than two options. This is done using the \textit{else if} 
            statement.

            \begin{lstlisting}
        if (temperature > 20) {
            console.log("It is hot");
        } else if (temperature > 15) {
            console.log("It is moderate");
        } else {
            console.log("It is cold");
        }
            \end{lstlisting}

            In this example, the statement will find the first true condition, and run 
            the respective code. In this example 17 degrees would return moderate, 
            instead of cold.
    
    \subsection{Logical operators}
        Sometimes it can make sense to check multiple conditions. This is done 
        using logical operators.
        
        \begin{center}
            \begin{tabular}{p{2.9cm}|p{4cm}|p{4.5cm}}
                \textbf{Symbol} & \textbf{Operation} & \textbf{Explanation} \\
                \hline
                \&\& & AND & Checks if two conditions are true. Returns true if both are.\\
                \hline
                $||$ & OR & Checks the state of two conditions. Returns true if one of the conditions are true.\\
            \end{tabular}
        \end{center}

    \clearpage % Temporary Formatting
        Usage examples:

        \begin{lstlisting}
        // AND ( && ) usage:
        if (temperature >= 22 && wind < 10) {
            console.log("It's beach weather!");
        }

        // OR ( || ) usage:
        if (raining === true || temperature < 18) {
            console.log("Remember a jacket")
        }
        \end{lstlisting}
    
    \subsection{\textbf{\textit{Coding practice: Conditional Execution}}}
        \begin{practice}
            \begin{enumerate}
                \item Code a small ”drawing program” where a circle is drawn constantly at the cursor’s location.
                \item Use the built in mouseIsPressed variable to print to the console if the mouse button is held down.
                \item Upgrade the program so the circles diameter increases when held down.
            \end{enumerate}
        \end{practice}
        \begin{practice}
            Figure out how key inputs work in p5.js, using p5js.org.
            Then draw a circle in the middle of the screen. When you press one of the arrow keys, the circle should move 
            in that direction.
        \end{practice}
        \begin{practice}
            Using a variable \textit{x}, draw a circle that moves along the x-axis on canvas with a specific speed. When the circle hits 
            a wall, it should change direction so it never leaves the canvas. 
            \\

            Hint: multiplying the speed with -1 will change the direction.
        \end{practice}
        \begin{practice}
            Make a canvas that can be divided into three equal vertical slices (for example 600x400). Make a program 
            that highlights one third of the canvas at a time with a rectangle, depending on whether the mouse is on 
            the left, middle or right third of the canvas.
        \end{practice}

    \clearpage % Temporary Formatting

\section{Loops and Arrays}
    Sometimes it is useful to repeat similar code multiple times. For example 
    drawing three circles with diameters 1, 2 and 3. We draw a circle three 
    times, but with varying diameters. It can be done manually like this:

    \begin{lstlisting}
        circle(x,y,1); // Circle with diameter = 1
        circle(x,y,2); // Diameter = 2
        circle(x,y,3); // Diameter = 3
    \end{lstlisting}

    The three lines of code are very similar and can be optimized using a 
    loop. In JavaScript we talk about two different types of loops. For-loops 
    and while loops. The difference is that in a for loop, we now the number
    of iterations.

    Try to think of the example with the circles. Should we use a for or 
    while loop to optimize the code?

    In this example the most optimal loop would be a for loop. It would, since we 
    have to draw a circle three times i.e. we have three iterations. 
    Note that a while loop also could be used for this example theoretically.

    \subsection{For loops}
        A for loop is used to repeat a block of code, a specific number of times.
        The structure for a for loop looks like this:
        \begin{lstlisting}
        for (let i = 0; i < 4; i++) {
            // code to repeat
        }
        \end{lstlisting}

        In the code above, we have a for loop repeating four times.
        First a variable \textit{i} is being declared, and assigned the value 0.
        Each time the code has run, the last argument in the for loop (\textit{i++}) runs.
        So in this example \textit{i} is being incremented by one, for each iteration of the code.

        If we add console.log(i) inside the for loop, like this:
        \begin{lstlisting}
        for (let i = 0; i < 4; i++) {
            console.log(i)
        }
        \end{lstlisting}

        And then run the code, the following will be printed to the console:

        \begin{lstlisting}
        0,
        1,
        2,
        3
        \end{lstlisting}

        Try the following thinking exercise:

        \begin{exercise}
            \begin{itemize}
                \item Why does it print 0,1,2,3 and not 1,2,3,4?
                \item What would the loop print if we changed the third argument from \textit{i++} to \textit{i+=2}?
                \item What would the loop print if we changed the first argument from \textit{let i = 0} to \textit{let i = 2}
            \end{itemize} %0,
        \end{exercise}

        Now that we now a bit about for loops, we can automize the example 
        with three circles from before, like this:

        \begin{lstlisting}
        for (let d = 1; d <= 3; d++) {
            circle(x,y,d);
        }
        \end{lstlisting}

        (x,y) could be any chosen coordinates.

        In this for-loop we use a variable d, that corresponds to the circles diameter.
        The for loop has three iterations, that look like this:

        \begin{description}
            \item[d = 1: ] circle(x, y, 1);
            \item[d = 2: ] circle(x, y, 2);
            \item[d = 3: ] circle(x, y, 3);
        \end{description}

        As we can see, it produces the same output as the manual example.

    \subsection{While loops}
        A while loop will repeat a specific block of code a unspecified number of times.
        The structure of a while loop, could look like this:

        \begin{lstlisting}
        while(condition) {
            // code to repeat
        }
        \end{lstlisting}

        For loops can be used instead of while loops in most scenarios. The primary
        use of while loops are when you don't know the number of iterations. For example
        when generating a random number like this:

        \begin{lstlisting}
        let i = 0;

        while(i !== 9) {
            i = random(0,9);
        }
        \end{lstlisting}

        This (completely useless) code, will first define the variable \textit{i} and
        then continue asssigning it a random value between 0 and 9, until the value 
        randomly becomes 9.
        
        Note that this loop would never actually stop, since the random() function produces floating-point numbers,
        so it would for example return a number like 1,2312355... We're waiting for \textit{i} to become the integer 9, which will never happen.
        
        The code would produce a infinite loop, where the condition will stay true indefinitely, which
        would ultimately crash your program.

        Try to draw the same three circles as drawn last, this time using a while loop.

        \begin{exercise}
            \begin{itemize}
                \item Create a while loop that draws three circles with the diameters 1,2 and 3. 
            \end{itemize}
        \end{exercise}

    \subsection{Arrays}
        Sometimes it is useful to store multiple values to one variable.
        Think of a graph containing multiple grades from a class. Instead of creating a new
        variable for each grade, we can instead create one variable containing all grades.

        \begin{lstlisting}
        let grades = [86, 92, 100, 54, 32, 54, 95, 
                                    56, 76, ...];
        \end{lstlisting}

        This is called an array. It's defined using square brackets: [ ]. Each data point inside the brackets
        are separated by commas.

        Each grade inside the grades array have an index. Note that the indexes start at 0 and not 1.
        So for example the grade 86 is at index 0, grade 92 is at index 1, grade 100 is at index 2, and so on.
        Getting one of the grades can be done using \textit{grades[i]} where \textit{i} is the index.
        For example \textit{grades[5]} would return 54.

        \begin{lstlisting}
        let grades = [86, 92, 100, 54, 32, 54, 95, 
                        56, 76, 77, 14, 33, 86, 85, 
                        96, 100, 60, 75];

        let sum = 0;

        for (let i = 0; i < grades.length; i++) {
            sum += grades[i];
        }

        console.log("The average grade is: ", 
                    sum / grades.length)
        \end{lstlisting}

        Output: "The average grade is: 70". 
        
        The code runs through all the grades and adds them together.
        When we have the sum of all grades we can divide them by the number of observations, which is the length
        of the array grades, to get the average grade.

        The code is responsive, so adding new grades to the array, will also update the output(average grade).

        \subsubsection{Modify arrays using simple array methods}
            We can manipulate an array while the code is running, using differnt methods:

            \begin{center}
                \begin{tabular}{p{4.5cm}|p{6.7cm}}
                    \textbf{Method} & \textbf{Explanation} \\ [0.5ex]
                    \hline
                    Array.push(element) & Adds element to the end of an array \\
                    \hline
                    Array.pop() & Removes the last element of an array \\
                    \hline
                    \\
                    
                    Array.unshift(element) & Adds element to the front of an array \\
                    \hline
                    Array.shift() & Removes the first element of an array \\
                \end{tabular}
            \end{center}

    \subsection{\textbf{\textit{Coding practice: Loops and Arrays}}}
        \begin{practice}
            Draw 100 circles at random locations on canvas, using a for loop. Then do it 
            with a while loop.
        \end{practice}
        \begin{practice}
            Use a for loop to draw all numbers between 0 and 100 on canvas on random 
            locations. The number should be blue if it’s even, and red if it’s uneven.
            
            Hint: use i\%2
        \end{practice}
        \begin{practice}
            Define an array “spanishGreetings” and give it these values: “Hola”, “Buenos 
            dias”, “Buenas tardes” and “Buenas noches”. Now create a for loop that runs 
            through all elements of the array and writes the greeting in a list downwards. 
            Add the value “Mucho gusto” to the array, and check if it’s written under the 
            previous element.
        \end{practice}
        \begin{practice}
            Create a 8x8 chess board on a 800x800 canvas. (changing between black and 
            white rectangles). Use a for loop nested in another for loop.
        \end{practice}

\clearpage % Temporary Formatting
\section{Functions}
    In this chapter we learn about functions - one of the most important concepts of 
    programming.

    Functions allow us to:
    \begin{enumerate}
        \item Organize code and structure large programs
        \item Avoid repetition by creating reusable logic
        \item Build more advanced programs
    \end{enumerate}

    \subsection{What is a function?}
        A function is a reusable block of code that performs a specific task.

        A function can take inputs, and return outputs based on your inputs.
        The basic syntax for a function looks like this:

        \begin{lstlisting}
        function functionName(inputs) {
            // code to run
        }
        \end{lstlisting}

        An example of a simple function, that does the same thing everytime, 
        ergo, does not take an input, may look like this:

        \begin{lstlisting}
        function sayHello() {
            console.log("Hello!");
        }
        \end{lstlisting}

        The function can be run (called) anywhere in the code, using: \textit{sayHello()}

        If you don't call the function, it won't run.

    \subsection{Funtion parameters}
        Functions can also be used as some kind of a "template" if you pass inputs(parameters) to it. 
        Parameters work as a kind of variable used inside functions. You can pass them on to the function, where
        they can be used to produce unique outputs.  
        This is a simple example, where we updated the \textit{sayHello} function, so it can take a parameter; \textit{word}:
        \begin{lstlisting}
        function say(word) {
            console.log(word);
        }
        \end{lstlisting}
        
        Now when calling the function \textit{say}, we can pass on an arbitrary value to the function, that it will print.
        For example calling \textit{say("Hello!")} would do the same as calling \textit{sayHello()}.
        We could also pass on other words to the \textit{say} function, for example \textit{say("¡Buenos Dias!")}, which would print
        "¡Buenos Dias!". Note that the system does not know what the input \textit{word} means. \textit{word}
        could also be replaced by another name like \textit{greeting} or something abstract like \textit{x}, 
        without changing what the function does.

        \begin{lstlisting}
        function greet(name) {
            console.log("Hello " + name);
        }

        greet("Anna"); // Prints: "Hello Anna"
        greet("Jonas"); // Prints: "Hello Jonas"
        \end{lstlisting}

        Try this short exercise.

        \begin{exercise}
            \begin{enumerate}[label=\alph*)]
                \item What is the parameter being passed on to the \textit{greet} function?
                \item Create the code for a function that takes the parameter \textit{age}, and then prints "Your age is ..." to the console.
            \end{enumerate}
        \end{exercise}
        
    \subsection{Function outputs}
        Other than being able to do small tasks like printing things to the console,
        functions can also be used to do more advanced operations. We can pass more
        parameters to the function, and return a output based on the inputs.

        \begin{lstlisting}
        function sum(a, b) {
            return a + b;
        }
        \end{lstlisting}

        Now when calling the function using two values, for example: \textit{sum(3,5)},
        it will return the sum of the inputs. So, passing 3 and 5, would return 8.
        
        \begin{lstlisting}
        console.log(10+sum(3,5)); 

        // Prints 18 to the console.
        \end{lstlisting}

        We can imagine how powerful the function could be if we use an array as a parameter. 
        If we think of the grading example from section 4.3, we can make a function that 
        calculates the average grade easily. We're gonna use a shortened version of the \textit{grades}
        array for simplification.

        \begin{lstlisting}
    let grades = [86, 92, 100, 54];

    function calculateAverageGrade(gradeArray) {
        // We take the parameter gradeArray. 
        // Note that this could be any arbitrary 
           array with grades.

        let sum = 0;

        for(let i = 0; i < gradeArray.length; i++) {
            sum += gradeArray[i]; 
            // This for-loop calculates the sum of 
               all elements in gradeArray
        }

        return sum / gradeArray.length;
    }

    let averageGrade = calculateAverageGrade(grades);

    console.log("The classes average grade is: " 
                    + averageGrade);

        \end{lstlisting}
    
        Note that variables defined inside functions, e.g. \textit{gradeArray}, are only
        accesible inside the function.
    
    \subsection{\textbf{\textit{Coding practice: Functions}}}

\section{Mini projects}
    When you feel confident in your basic coding skills, improve them by using them in practice.
    We made a couple of small projects that should help you use different parts of programming to 
    make larger and more diverse programs. The projects gradually increase in difficulty.
    
    \begin{project}[Reaction Clicker Game]
        Create a simple reaction game, where a red circle appears at a random 
        location on the canvas. When the player clicks the circle, they 
        gain 1 point, kept in a “score” variable. The circle (after being clicked) 
        moves to a new random position and it can then be clicked again. Display 
        the “score” variable on screen.
    \end{project}
    \begin{project}[Falling Objects Dodging Game]
        Create a game where a player controls a blue circle at the bottom of the 
        canvas using arrow keys. Red circles randomly fall from the top. If a red circle 
        hits the player a “Game Over” screen appears. Keep track of the time survived 
        (can be frames).
    \end{project}
    \begin{project}[Two-player Competitive Target Game]
        Make a simple two-player game, where two players compete to catch targets. One 
        player (a white circle) can move around canvas with the WASD-keys, the second 
        player (a blue circle) can move around canvas with the arrow keys. The players 
        shouldn’t be able to go outside the canvas. At the start of each game, a small 
        red circle (the target) should appear at a random position. If one of the players 
        touches the target, it will disappear and a new one will appear at a new random 
        position. The player that catch the target will get a point.

        When you have the base game, add your own innovative feature. It could be moving
        targets, obstacles on the canvas or anything else!

    \end{project}

\section{Manage Your Projects Using GitHub}
    \begin{center}
        \textbf{Creating a GitHub Account}
    \end{center}

    When working on bigger projects over longer periods of time, whether alone or in groups, it can be quite beneficial to document your development process and keep track of previous versions of your work. This is where Git and GitHub become very useful tools.
    Git is a version control system that helps you track changes in your project over time, whilst GitHub is an online platform that hosts Git repositories in the cloud.
    You can complete this book without using GitHub. However, if you are planning to:

    \begin{itemize}
        \item Work on longer-term creative code projects
        \item Collaborate with others
        \item Build a portfolio
        \item Share your code online
        \item Safely experiment without the fear of breaking functional code
    \end{itemize}

    Then learning the basics of GitHub will make your coding experience and workflow much smoother. If you already have a GitHub account and have experience using it for other projects, you may still want help with setting it up in combination with VS Code. 
    \begin{center}
        \textbf{Getting Started}
    \end{center}
    
    \paragraph{Git and GitHub} 
        To get started, you will need to make a GitHub account. This is done by going to GitHub’s website at: https://github.com/ and signing up. Once you have created your account, you are ready to connect it to VS Code.
        The next step is to check whether you already have Git installed in VS Code, as that is what is used for version control at base. To do so, all you need to do is:

        \begin{enumerate}
            \item Open VS Code
            \item Open the terminal (under “View” select the option “Terminal”)
            \item In the terminal, type “git --version” and press enter. 
        \end{enumerate}

        If Git is not installed, you can download it from: https://git-scm.com/.
        Install it using default settings.

    \paragraph{Connecting GitHub to VS Code}
        Now that you have created a GitHub account and made sure that Git is installed, it is time to connect it all together in VS Code. To do so, follow the steps below:

        \begin{enumerate}
            \item Open VS Code
            \item Click the Accounts tab
            \item Choose “Sign in with GitHub”
            \item Follow the browser authentication steps
        \end{enumerate}

        Once you’ve authenticated successfully, your VS Code will now be linked to your GitHub account. 

    \paragraph{Creating a Git repository}
        Let’s say you’ve already created your p5 project folder. To initialize Git, all you need to do is:

        \begin{enumerate}
            \item Open your project folder in VS Code via the explorer tab or otherwise
            \item Click the Source Control tab
            \item Click “Initialize Repository”
        \end{enumerate}

        Once the repository is initialized, VS Code will start tracking changes in your folder.

    \paragraph{Committing to GitHub}
        When you make changes to your project, you will be able to save these changes as a ‘commit’. A commit is a snapshot of your project at a specific moment. To make a commit is actually very simple. After initializing the repository in the previous step, you will be able to view changes made to your project under the Source Control tab, specifically under “Changes”. Here you will see the specific files that have been modified. 
        To commit a change, hover over the file in the opened tab, and press the plus icon. As soon as you do so, the file will be moved up to a new tab called “Staged Changes”. The last, and often the most crucial step of the process, is writing a short and concise message describing the changes made. This may seem trivial now, but your future self will thank you for it. Once you are done noting the changes, you can push the commit to GitHub by clicking on the arrow next to the “Commit” button and selecting “Commit and Push” from the dropdown menu.

    \paragraph{Useful Extensions}
        To make your experience using Git and GitHub in VS Code smoother, we recommend you get a few extensions installed.

        \subparagraph{Git Graph (by mhutchie)}
            This extension provides a visual representation of your Git repository’s history directly inside VS Code. Instead of reading commits as a simple list, Git Graph also shows you branches, merges, tags, and the overall project timeline. 
            This makes Git easier to understand visually and simplifies working with branches as well as seeing how different versions of your project connect.

        \subparagraph{Git History (by Don Jayamanne)}
            This extension allows you to explore the full change history of your files and projects within VS Code. With it, you can:

            \begin{itemize}
                \item View previous versions of a file
                \item Compare changes between commits
                \item See who made specific changes (in collaborative projects)
                \item Inspect detailed commit logs
            \end{itemize}

            This makes debugging easier and helps you understand better how your project evolved.

    \paragraph{The Basic Workflow}
        Using GitHub can seem a bit daunting, but the workflow gets simplified and usually goes a little something like this:

        \begin{enumerate}
            \item Make changes to your project
            \item Save the files
            \item Review changes in Source Control
            \item Note changes made to the files
            \item Commit and push changes to GitHub
            \item Repeat steps 1-5
        \end{enumerate}

    \paragraph{Working With other people}
        Using Git and GitHub for group projects is straightforward and highly effective. Once one group member has created a repository, they can add the other members as collaborators in the repository settings on GitHub. 
        Each collaborator must then: 

        \begin{enumerate}
            \item Copy the repository link from GitHub
            \item Use “Clone Repository” in VS Code under the Source Control tab
            \item Download a local copy of the project to their computer
        \end{enumerate}

        Cloning the repository creates a full working copy of the shared project on each person’s machine. From here, each member can:

        \begin{itemize}
            \item Make changes locally
            \item Commit their changes (create a snapshot of the project)
            \item Push changes to the shared GitHub repository
        \end{itemize}

        Before continuing work, members should also pull changes from GitHub to ensure they are working on the latest version of the project.

       
\clearpage
\section{Attachments}

    \subsection{Summary list for exercises}
    \textbf{Exercise 2.1:}
        \begin{enumerate}[label=\alph*)]
            \item
                \begin{tabular}{c | c}
                    number(float) & output = 198.5
                \end{tabular}
            \item
                \begin{tabular}{c | c}
                    string & output = “180.5erasmus”
                \end{tabular}
            \item
                \begin{tabular}{c | c}
                    string & output = “erasmusfalse”
                \end{tabular}
        \end{enumerate}
    

    \noindent\textbf{Exercise 4.1:}
        \begin{enumerate}
            \item 
            Possible answer: the variable \textit{i} starts at 0. The value will be 
            incremented by 1 as long as \textit{i} is lesser than 4. So in the first 
            iteration we have i = 0. The code runs and then \textit{i++} is run. \textit{i} is now
            equal to one, and the same continues all the way up to three, which is the last integer that's lesser than 4.
            \item 0, 2
            \item 2, 3
        \end{enumerate}

    \noindent\textbf{Exercise 4.2:}
        \begin{lstlisting}
    let d = 1;
    
    while(d <= 3) {
        circle(x,y,d);
    }
        \end{lstlisting}

    \noindent\textbf{Exercise 5.1:}
        \begin{enumerate}[label=\alph*)]
            \item \textit{name}
            \item Possible answer: 
            \begin{lstlisting}
function printAge(age) {
    console.log("Your age is " + age);
}

printAge(21); // Prints: "Your age is 21"
            \end{lstlisting}
        \end{enumerate}

\end{document}